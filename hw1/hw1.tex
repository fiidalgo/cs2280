\documentclass[11pt]{article}
\usepackage[margin=2.5cm]{geometry}
\usepackage{parskip}
\usepackage{microtype}
\usepackage{enumitem}
\usepackage{amsmath, amsfonts, amssymb, amsthm}
\usepackage{thmtools}
\usepackage{bbm}

\declaretheorem{theorem}
\declaretheorem[style=definition]{problem}
\declaretheorem[style=remark, numbered=no]{hint}
\declaretheorem[style=remark, numbered=no]{note}

\newcommand*{\C}{{\mathcal C}}
\newcommand*{\Hp}{{\mathcal H}}
\newcommand*{\R}{\mathbb{R}}
\newcommand*{\Z}{\mathbb{Z}}

\DeclareMathOperator{\E}{\mathbb{E}}
\DeclareMathOperator{\1}{\mathbbm{1}}
\DeclareMathOperator{\Ber}{Ber}


\begin{document}

\begin{flushright}
			Leslie G. Valiant \\
			TFs: Aayush Karan and Kevin Cong
\end{flushright}

\begin{center}
\textbf{CS 2280: Computational Learning Theory} \\
\textbf{Homework 1}\\
\textbf{Due: Feb. 19, 11:59pm}\\
\textbf{Nico Fidalgo}
\end{center}

\paragraph{Note} For some problems you may need the following fact from probability theory called the Chernoff bound:

\begin{theorem}[Chernoff bound] Let $X_i$, $1 \leq i \leq m$, be independent Bernoulli random variables each with probability of success $\mu$. Then for any $\lambda \in [0,1]$, 
	\[ \Pr \left[ \left| \frac{1}{m} \sum_{i=1}^m X_i - \mu\right| \geq \lambda \mu
	\right] \leq 2 e^{-\lambda^2 \mu m /2} \]
\end{theorem}

You can find more details and generalizations in the textbook or online.

\paragraph{Policy reminders} You are strongly encouraged to type your solutions using
\LaTeX. You may discuss problems with your classmates, but not merely copy each others solutions. You must write all solutions by yourself, list your collaborators on your problem sets and also appropriately cite any resources outside of the class materials that you have used. You are not allowed to look up solutions to the problems. Please do not use LLMs or LLM-assisted tools for finding solutions to the problems.

\rule{\linewidth}{0.4pt}

\begin{problem}(10pt) \textbf{Learning union of intervals.} Give a PAC-learning algorithm for the concept class $\mathcal{C}_{2}$ formed by unions of two closed intervals, that is $[a, b] \cup[c, d]$, with $a, b, c, d \in \mathbb{R}$. Extend your result to derive a PAC-learning algorithm for the concept class $\mathcal{C}_{n}$ formed by unions of $n \geq 1$ closed intervals, thus $\left[a_{1}, b_{1}\right] \cup \cdots \cup\left[a_{n}, b_{n}\right]$, with $a_{i}, b_{i} \in \mathbb{R}$ for $k \in[n]$. What are the time and sample complexities of your algorithm as a function of $n$ ?
\end{problem}
To give a PAC-learning algorithm for learning the union of \(n\) intervals, the first step is to define the learning model and hypothesis class. The concept is the union of \(n\) intervals, and a hypothesis class should predict \(1\) if a point is in any of the learned intervals and 0 otherwise with error at most \(\epsilon\) with probability of successfully finding a good hypothesis at least \(1-\delta\).\\
Given a set of positive samples \( S= \{ x_1, x_2, \ldots, x_m \} \), we determine the intervals by checking which pairs of positive samples are adjacent. Assume negative samples between each interval have been drawn for the purpose of separating the intervals: \(\{c_0, c_1, \ldots, c_n \}\). Our hypothesis is therefore defined as: \(h(x)=1\) if \(x \in [x_i, x_j]\) for some \(x_i, x_j \in S\) and that there is no overlap between \([x_i, x_j]\) and the negative samples, i.e., \([x_i, x_j] \cap \{c_0, c_1, \ldots, c_n \} = \emptyset\), otherwise \(h(x)=0\).\\
To ensure the error rate is below \(\epsilon\), we need enough positive samples in every interval so that we do not miss points near the boundaries. Error occurs if we fail to get enough positive samples in the extreme \(\frac{\epsilon}{2}\) tails of each interval. The probability of missing any of these \(2n\) tails is given by: \(P(\text{miss any of } 2n \text{ tails}) \leq 2nP(\text{miss one tail})\) where: \(P(\text{miss one tail})=(1-\frac{\epsilon}{2})^m \leq e^{-\frac{m\epsilon}{2}}\) by standard exponential bound. Thus, \(2ne^{-\frac{m\epsilon}{2}} \leq \delta\), and by taking the natural logarithm of both sides, we get: \(\frac{2}{\epsilon}\ln \frac{2n}{\delta} \leq m\). Therefore, the sample complexity for positive samples is logarithmic in \(n\), i.e., \(O(\log n)\). The sample complexity of negative samples is \(O(n)\) as we need at most \(n\) negative samples to ensure every gap between two intervals is correctly identified. The time complexity of processing \(m\) positive samples and \(n\) negative samples is \(O(m + n)\), but as \(m\) grows logarithmically in \(n\), the overall time complexity is \(O(\log n + n)=O(n)\)



\begin{problem} (10pt) \textbf{PAC-learning parity functions.}
	\emph{Parity functions} is analogous to conjunctions or disjunctions, but with the XOR operation $\oplus$ rather than $\wedge$ or $\vee$. To be precise, given a subset $S \subseteq \{1, 2, \dots, n\}$, a parity function $f_S: \{0,1\}^n \to \{0,1\}$ outputs 1 iff the number of 1s in the subset of the input indexed by $S$ is odd. For example, if $n = 4$ and $S = \{1,3,4\}$, the parity function $f_S = x_1 \oplus x_3 \oplus x_4$ will return $f(0010) = 1, f(1010) = 0$. The class of parity functions $\mathcal{P}$ includes all functions that can be described in this way:
	\[ \mathcal{P} = \{f_S: S \in 2^{\{1,\dots,n\}}\} \qquad
	f_S(x) = \bigoplus_{i \in S}x_i \]
	Show that $\mathcal{P}$ is PAC learnable.
\end{problem}
\begin{hint}
	Note that $x \oplus y$ is equal to $x + y \pmod 2$. Additionally, searching for an Occam algorithm might be easier than working with $\epsilon$ and $\delta$.
\end{hint}




\begin{problem} (10pt) \textbf{``Relaxed'' $\delta$-dependence in PAC definition.}
Prove that if a concept class $\C$ is PAC-learnable by class $\Hp$ by the usual definition (in time polynomial in $1/\delta, 1/\epsilon, n, \text{size}(c)$), then it is in fact learnable by class $\Hp$ in time polynomial in $\log(1/\delta), 1/\epsilon, n, \text{size}(c)$. (Notice the more stringent requirement on the
running time's dependence on $\delta$.)
\end{problem}
\begin{hint}
	Can we boost our confidence by running the original PAC-learning algorithm multiple times?
\end{hint}



\begin{problem} (10pt) \textbf{Approximate Occam algorithm.}
For constants $\alpha \geq 1$, $\beta <1$, and $\gamma
<1/2$, let a $\gamma$-approximate $(\alpha,\beta)$-Occam algorithm be the same
as an $(\alpha,\beta)$-Occam algorithm, but instead of consistency with all the
examples only consistency with at least $1-\gamma$ fraction of examples is
guaranteed. Similarly let a $\gamma$-PAC learning algorithm for a concept class
$\C$ be the same as a PAC learning algorithm except that it produces a
$(\gamma+\epsilon)$-approximate (instead of an $\epsilon$-approximate)
hypothesis.

Show that if there is a $\gamma$-approximate $(\alpha,\beta)$ Occam
algorithm for $\C$ then there is a $\gamma$-PAC learning algorithm for $C$.
\end{problem}


%%%%%% PROBLEM %%%%%%
\begin{problem}(15pt) \textbf{Learning unions of halfspaces is NP-hard.} A {\em linear threshold function} or a {\em halfspace} is a function
$f(x): \{0,1\}^n \rightarrow \{0,1\}$ such that there exists some
$n$-dimensional real vector $w$ and real number $\theta$ such that
$f(x) = 1$ iff $w \cdot x \geq \theta.$ 
%A linear threshold function is often called a {\em halfspace}. 
A union of halfspaces is simply a disjunction of several halfspaces. Suppose that for some $k \ge 3$, unions of $k$ halfspaces are
efficiently PAC learnable by an algorithms that outputs unions of $k$ halfspaces. 
Show that then there is a randomized algorithm $R$ taking as input a graph
$G$ that runs in
time polynomial in the size of $G$ with the
following properties:
\begin{itemize}
\item If $G$ is $k$-colorable then with probability
at least $2/3$, $R$ outputs a $k$-coloring of $G$.
\item If $G$ is not $k$-colorable, then $R$'s output may be arbitrary.
\end{itemize}

\hint Consider adding the example $\langle 1^n, - \rangle$ to the reduction you saw in class.
\end{problem}


%%%%%% PROBLEM %%%%%%

\begin{problem}(15pt) \textbf{Conjunction of PAC learnable classes is PAC learnable.}  
Show that if both class $C_1$ and class $C_2$ are PAC learnable from
positive examples only (each separately), then class $C=\{c_1\wedge c_2\ |\
c_1\in C_1\mbox{ and } c_2\in C_2\}$ is also PAC learnable from positive
examples only.
\end{problem} 
\begin{hint}
Let $h = h_1 \wedge h_2$, where $h_1$, $h_2$ are obtained from feeding the positive 
examples of $c_1\wedge c_2$. What do we know about the relationship between $c_1$
and $h_1$?
\end{hint}


%%%%%% PROBLEM %%%%%%

\begin{problem} (15pt) \textbf{Probabilistic PAC is equivalent to deterministic PAC.} Let $\C$ be any concept class and $\Hp$ be any hypothesis class. Let $h_0$ and
$h_1$ be representations of the identically $0$ and identically $1$ functions
respectively. Show that if there is a randomized algorithm for efficiently PAC
learning $\C$ using $\Hp$, then there is a deterministic algorithm for
efficiently PAC learning $\C$ using $\Hp \cup \{h_0, h_1 \}$. 

\emph{Note}: There are two sources of randomness we are talking about. \\
1. The inherent randomness that occurs because the example oracle EX is random.
\\
2. The randomization that the algorithm uses. You are asked to get rid of the
second randomization. More precisely if you run the learning algorithm several
times using the same sequence of examples as output by the example oracle, the
final hypothesis your output should be the same (and almost accurate with high
probability).

This problem is complex, so we will help you organize the solution as below. 
Observe that a randomized algorithm can be viewed as a deterministic
algorithm which takes as additional input a random string, which we may regard
as a sequence of fair coin toss. The algorithm is then
required to work correctly with high probability over this input string. 
Let the randomized algorithm be $L$ and consider the following deterministic 
algorithm:
\begin{center}
\fbox{ 
\begin{minipage}{0.8 \textwidth} 
%\texttt{// M will be defined in the text} \\
Using an oracle $EX(c, \mathcal{D})$ get $M$ data points. Set $i = 1$.  \\
If less than $\epsilon M/2$ examples have label $0$ return $h_1$. \\
Else if less than $\epsilon M/2$ examples have label $1$ return $h_0$. \\
Else 
\begin{itemize} 
\item Whenever $L$ requests an example return $(x_i, y_i)$ and increment $i$. \\
\item Whenever $L$ requests a random bit as follows:
\begin{itemize}
\item If $y_i \ne y_{i+1}$, then return $y_i$ and increment $i$ twice. 
\item Otherwise, increment $i$ twice, effectively discarding these two points.
\end{itemize}
If $L$ terminates, return the hypothesis output by $L$. 
Else whenever $i>M$, i.e. we run out of data points, output $h_0$. 
\end{itemize}
\end{minipage}}
\end{center}

There are now three sources of error from this deterministic algorithm:
\begin{itemize}
\item Incorrectly choosing $h_0$ or $h_1$ as the hypothesis at the start;
\item $L$ terminates but returns a hypothesis with a large error;
\item $L$ fails to terminate before running out of $M$ examples. 
\end{itemize}

For simplicity, you may assume that $L$ requires $m$ examples and $t$ random bits
to output with probability at least $1-\delta/3$ a hypothesis with error at most
$\epsilon$. Note that in reality $m$ and $t$ are not necessarily fixed, but in order for $L$ 
to be a PAC-learning algorithm, they must be polynomial in all the relevant parameters.
Therefore, the second source of error is taken care of. 

\begin{enumerate}
\item (3pt) Explain the algorithm for generating a random bit above. Why does it produce a fair coin toss? 
\item (5pt) Find a lower bound on $M$ such that if we choose $h_0$ or $h_1$ as the 
final hypothesis at the start, then with probability at least $1-\delta/3$, we
will have error at most $\epsilon$.   
\item (7pt) Find a lower bound on $M$ such that assuming we do not choose $h_0$ or $h_1$ as the final hypothesis at the start, with probability at least $1-\delta/3$ we will generate at least $t$ random bits using $M-m$ examples. 
\end{enumerate}
\end{problem}

\end{document}