\documentclass[11pt]{article}
\usepackage[margin=2.5cm]{geometry}
\usepackage{parskip}
\usepackage{microtype}
\usepackage{enumitem}
\usepackage{amsmath, amsfonts, amssymb, amsthm}
\usepackage{thmtools}
\usepackage{bbm}

\declaretheorem{theorem}
\declaretheorem[style=definition]{problem}
\declaretheorem[style=remark, numbered=no]{hint}
\declaretheorem[style=remark, numbered=no]{note}

\newcommand*{\C}{{\mathcal C}}
\newcommand*{\Hp}{{\mathcal H}}
\newcommand*{\R}{\mathbb{R}}
\newcommand*{\Z}{\mathbb{Z}}

\DeclareMathOperator{\E}{\mathbb{E}}
\DeclareMathOperator{\1}{\mathbbm{1}}
\DeclareMathOperator{\Ber}{Ber}
\DeclareMathOperator{\size}{size}
\DeclareMathOperator{\VCD}{VCD}


\begin{document}

\begin{flushright}
			Leslie G. Valiant \\
			TFs: Aayush Karan \& Kevin Cong
\end{flushright}

\begin{center}
\textbf{CS 228: Computational Learning Theory} \\
\textbf{Homework 2}\\ \textbf{Due: March 12, 11:59PM}
\end{center}

\textbf{Policy reminders:} You are strongly encouraged to type your solutions using LATEX. You may
discuss problems with your classmates, but not merely copy each others solutions. You must write
all solutions by yourself, list your collaborators on your problem sets and also appropriately cite
any resources outside of the class materials that you have used. You are not allowed to look up
solutions to the problems. Please do not use LLMs or LLM-assisted tools for finding solutions to the
problems. 
\rule{\linewidth}{0.4pt}

\begin{problem} [10pts] \textbf{Efficient Evaluability is Necessary.}
The definition of learning in the PAC model requires that hypotheses produced by a learning algorithms are evaluatable efficiently (that is, in polynomial time). In this problem you will show that this requirement is necessary for a meaningful notion of learning.

Suppose we consider relaxing this restriction, and let $\Hp$ be the class of all Turing machines (not necessarily polynomial time). Show that if $\C_n$ is the class of all Boolean circuits of size at most $p(n)$ for some fixed polynomial $p(\cdot)$, then $\C$ is efficiently PAC learnable using $\Hp$.

Argue that your solution shows that this relaxation trivializes the model of learning. (Hint: How could you use your solution to trivially learn other concept classes we've seen in class?)

To show that the requirement that the hypotheses produced by a learning algorithm are efficently evaluatable in polynomial time, consider the following scenario: Suppose we have a learning algorithm $\mathcal{L}$ that efficiently PAC learns the concept class $\mathcal{C}_n$ of Boolean circuits with size bounded by $p(n)$:

\textbf{Algorithm $\mathcal{L}$:}
\begin{itemize}
	\item \textbf{Input:} A sample $S = \{(x_1, y_1), \ldots, (x_m, y_m)\}$ where each $y_i = c(x_i)$ for some target concept $c \in \mathcal{C}_n$
	\item \textbf{Output:} A Turing machine $T_S$ that implements a hypothesis $h$
\end{itemize}

\textbf{Definition of $T_S$:} For any input $x \in \{0,1\}^n$, $T_S$ executes:
\begin{enumerate}
	\item Systematically enumerate all Boolean circuits $c' \in \mathcal{C}_n$ in order of increasing circuit size
	\item For each circuit $c'$, check if $c'(x_i)=y_i$ for all $(x_i, y_i) \in S$
	\item Return $c'(x)$ for the first circuit $c'$ is found that is consistent with $S$
\end{enumerate}

\textbf{Proof $\mathcal{L}$ is an Occam Algorithm:} Since the target concept $c \in \mathcal{C}_n$ is consistent with $S$, the algorithm will find at least one consistent circuit (possibly $c$ itself or another circuit that agrees with $c$ on all examples in $S$). Also, the description size of $T_S$ is polynomial in $n$ and $m$. Specifically, the Turing machine $T_S$ can be described in $O(n + \log m)$ bits as it only has to encode the enumeration procedure for circuits (constant-sized code), the sample $S$ (requires at most $O(m \cdot n)$ bits), and the consistency checking logic (constant-sized code). Thus, for some constants $\alpha$ and $\beta$, we have $\text{size}(T_S) \leq (n \cdot \text{size}(c))^\alpha \cdot m^\beta$. Lastly, constructing $T_S$ takes polynomial time since we only need to encode the sample and search procedure. Therefore, $\mathcal{L}$ is an Occam algorithm that efficiently PAC learns $\mathcal{C}_n$ using $\mathcal{H}$ as the hypothesis class.

\textbf{Conclusion:} This approach can be generalized to learn \textit{any} concept class that has a finite representation. For any learnable concept class $\mathcal{D}$, we can define a similar Turing machine that enumerates all concepts in $\mathcal{D}$ and returns the first one consistent with the sample. Despite such a Turing machine having a compact description making the training process efficient, evaluating it on new examples would be computationally expensive. The Turing machine could have to do an exhaustive search through an exponentially large search space of possible concepts before producing an output. Thus, without the efficiently evaluatable constraint, we could trivially learn any concept class by outputting a hypothesis that performs a brute-force search through the concept space-a strategy that doesn't reflect the true nature of learning.

\end{problem}
\begin{problem} [15pts] \textbf{Infinite Mistake-Bounded Model.}
	The \textit{Infinite Mistake-Bounded (IMB) model} is a Boolean on-line learning model where the number
of attributes in the world may be infinite. It is assumed, however, that each example has a finite
number of attributes assigned 1. An example is presented to the learner as a list of its positive
attributes. The learning scenario is the same as in the (standard) mistake-bounded model.

Provide an on-line learning algorithm that makes at most $\mathcal O(k \log n)$ mistakes when learning any
monotone disjunction of $k$ literals, if every example presented to the learner has at most $n$ positive
attributes.

\textit{Note: Assume for full credit that the learner does \textit{not} know the value of $n$ in advance. For partial credit, you can use the value of $n$ in your algorithm}

\textbf{Algorithm $\mathcal{L}$:}
\begin{itemize}
	\item Initialize $S = \emptyset$ (set of observed attributes)
	\item For each attribute $i \in S$, initialize weight $w_i = 1$
	\item Initialize threshold $\theta = 1$
	\item For each example $x$: \begin{itemize}
		\item If any new attributes appear in $x$, add them to $S$ and initialize their weights to 1
		\item Compute the sum of weights: $\text{sum} = \sum_{i \in x} w_i$
		\item Predict 1 if $\text{sum} \geq \theta$ and 0 otherwise
		\item If prediction is incorrect: \begin{itemize}
			\item If false negative (predict 0, true label of 1), then for each attribute $i \in x$ set $w_i = 2 \cdot w_i$
			\item If false positive (predict 1, true label of 0), then for each attribute $i \in x$ set $w_i = w_i/2$
		\end{itemize}
	\end{itemize}
\end{itemize}

Let $R$ be the set of $k$ relevant attributes in the target disjunction. In cases of false negatives, at least one of the relevant attributes must be present in the example. Each time a flase negative occurs, the weights of all attributes in the example (including at least one relevant attribute) are doubled. For any reelvant attribute $r \in R$, its weight can only doubled at most $\log n$ times before exceeding $n$ because the weight starts at 1, so after $\log n$ doublings, the weight becomes $2^{\log n} = n$ and since examples have at most $n$ positive attributes, the sum of weights cannot exceed $n^2$ without triggering a positive prediction. As there are $k$ relevant attributes each of which can be involved in at most $\log n$ false negative mistakes, the total contribution of mistakes due to false negatives is upper bounded by $k \log n$. For false positives, none of the relevant attributes are present, so for a false positive to occur, the sum of the weights of irrelevant attributes in the example must exceed the threshold. After each false positive mistake, the total weight of irrelevant attributes decreases by at least a factor of 2, so given that the maximum total weight is bounded (since each example has at most $n$ attributes), and the minimum possible weight of an attribute is arbitrarily small but positive, the number of false positive mistakes is also bounded by $O(k \log n)$. Therefore, the total mistake bound is $O(k \log n + k \log n) = O(k \log n)$. As we only maintain weights for attributes we've observed, we don't need to know $n$ in advance for this algorithm to work.
\end{problem}

\begin{problem} [15 points] \textbf{Function Classes: Linear Thresholds, DNFs, and Decision Lists.} \
	\begin{enumerate}[label=\alph*)]
		\item Prove that any 1-decision list over $x_1,\dots,x_n$ can be expressed as a linear threshold function $\1[w \cdot x \geq \theta]$, where $w \in \R^n\setminus\{\mathbf{0}\}$, $\theta \in \R$.
		\item Prove that the class of linear threshold functions and the class of
		poly$(n)$-term DNF are incomparable in that neither class is contained in the
		other.
	\end{enumerate}

\textbf{Part (a):} Define $z_j$ for $j=1,2,\dots,k$ as:
\begin{align}
	z_j = 
	\begin{cases}
	x_{i_j} & \text{if } b_j = 1 \\
	1 - x_{i_j} & \text{if } b_j = 0
	\end{cases}
	\end{align}

This transformation allows the decision list to be rewritten as:
\begin{align}
	&\text{if } z_1 = 1 \text{ then output } c_1 \text{ else} \\
	&\text{if } z_2 = 1 \text{ then output } c_2 \text{ else} \\
	&\vdots \\
	&\text{if } z_k = 1 \text{ then output } c_k \text{ else} \\
	&\text{output } c_{k+1}
	\end{align}

Let $M=2^k$ and define the weights as:
\begin{align}
	w_j = 
	\begin{cases}
	M \cdot 2^{k-j} & \text{if } c_j = 1 \\
	-M \cdot 2^{k-j} & \text{if } c_j = 0
	\end{cases}
	\end{align}

For the threshold, set: 
\begin{align}
	\theta = 
	\begin{cases}
	-\frac{1}{2} & \text{if } c_{k+1} = 1 \\
	\frac{1}{2} & \text{if } c_{k+1} = 0
	\end{cases}
	\end{align}

Substituting $z_j$, you get a standard linear threshold function on the original variable $x_i$:
\begin{align}
	\sum_{j=1}^k w_j z_j &= \sum_{j=1, b_j=1}^k w_j x_{i_j} + \sum_{j=1, b_j=0}^k w_j (1-x_{i_j}) \\
	&= \sum_{j=1, b_j=1}^k w_j x_{i_j} + \sum_{j=1, b_j=0}^k w_j - \sum_{j=1, b_j=0}^k w_j x_{i_j} \\
	&= \sum_{i=1}^n v_i x_i + v_0
	\end{align}

where: 
\begin{align}
	v_i &= \sum_{j: i_j=i, b_j=1} w_j - \sum_{j: i_j=i, b_j=0} w_j \\
	v_0 &= \sum_{j: b_j=0} w_j
	\end{align}

Thus, the final linear threshold function is $\1[v \cdot x + v_0 \geq \theta]$, or, $\1[v \cdot x \geq \theta - v_0]$.

\textbf{Part (b):} To prove that the class of linear threshold functions and the class of poly$(n)$-term DNF are incomparable in that neither class is contained in the other, I need to first show that there exists a linear threshold function that cannot be expressed as a poly$(n)$-term DNF, and then show that there exists a poly$(n)$-term DNF that cannot be expressed as a linear threshold function.

\textbf{Part 1:} A linear threshold function not representable as a poly$(n)$-term DNF.

Consider the Majority function $\text{Majority}_n(x)$ that outputs 1 if and only if at least $\frac{n}{2}$ variables are 1:
\begin{align}
\text{Majority}_n(x) = \1\left[\sum_{i=1}^n x_i \geq \frac{n}{2}\right]
\end{align}

This is clearly a linear threshold function with $w_i = 1$ for all $i$ and $\theta = \frac{n}{2}$.

To represent this as a DNF, we need a term for each possible way to set at least $\frac{n}{2}$ variables to 1. The number of such terms is:
\begin{align}
\sum_{i=n/2}^n \binom{n}{i}
\end{align}

For large $n$, this sum is dominated by the binomial coefficient $\binom{n}{n/2}$, which is approximately $\frac{2^n}{\sqrt{\pi n/2}}$ by Stirling's approximation. This grows faster than any polynomial in $n$.

Therefore, the Majority function cannot be represented by a poly$(n)$-term DNF.

\textbf{Part 2:} A poly$(n)$-term DNF not representable as a linear threshold function.

Consider the Parity function on 3 variables:
\begin{align}
\text{Parity}_3(x_1, x_2, x_3) = x_1 \oplus x_2 \oplus x_3
\end{align}

This can be expressed as a DNF with 4 terms:
\begin{align}
\text{Parity}_3(x_1, x_2, x_3) = (x_1 \land \lnot x_2 \land \lnot x_3) \lor (\lnot x_1 \land x_2 \land \lnot x_3) \lor (\lnot x_1 \land \lnot x_2 \land x_3) \lor (x_1 \land x_2 \land x_3)
\end{align}

To prove this cannot be represented as a linear threshold function, assume by contradiction that there exist weights $w_1, w_2, w_3$ and threshold $\theta$ such that:
\begin{align}
\text{Parity}_3(x_1, x_2, x_3) = \1[w_1 x_1 + w_2 x_2 + w_3 x_3 \geq \theta]
\end{align}

From the truth table of Parity$_3$, we have:
\begin{align}
&w_1 \cdot 1 + w_2 \cdot 0 + w_3 \cdot 0 \geq \theta \quad \text{(since Parity}_3(1,0,0) = 1\text{)} \\
&w_1 \cdot 0 + w_2 \cdot 1 + w_3 \cdot 0 \geq \theta \quad \text{(since Parity}_3(0,1,0) = 1\text{)} \\
&w_1 \cdot 0 + w_2 \cdot 0 + w_3 \cdot 1 \geq \theta \quad \text{(since Parity}_3(0,0,1) = 1\text{)} \\
&w_1 \cdot 1 + w_2 \cdot 1 + w_3 \cdot 1 \geq \theta \quad \text{(since Parity}_3(1,1,1) = 1\text{)} \\
&w_1 \cdot 0 + w_2 \cdot 0 + w_3 \cdot 0 < \theta \quad \text{(since Parity}_3(0,0,0) = 0\text{)} \\
&w_1 \cdot 1 + w_2 \cdot 1 + w_3 \cdot 0 < \theta \quad \text{(since Parity}_3(1,1,0) = 0\text{)} \\
&w_1 \cdot 1 + w_2 \cdot 0 + w_3 \cdot 1 < \theta \quad \text{(since Parity}_3(1,0,1) = 0\text{)} \\
&w_1 \cdot 0 + w_2 \cdot 1 + w_3 \cdot 1 < \theta \quad \text{(since Parity}_3(0,1,1) = 0\text{)}
\end{align}

From the first three inequalities, we have $w_1 \geq \theta$, $w_2 \geq \theta$, and $w_3 \geq \theta$.
But from the sixth inequality, we have $w_1 + w_2 < \theta$.

This creates a contradiction since $w_1 + w_2 \geq \theta + \theta = 2\theta > \theta$.

Therefore, Parity$_3$ cannot be represented as a linear threshold function, despite having a small DNF representation with only 4 terms.

Since I've shown examples in both directions, it can be concluded that the class of linear threshold functions and the class of poly$(n)$-term DNF are incomparable.
\end{problem}
\begin{problem}[15 points] \textbf{Perceptrons Can Make Exponentially Many Mistakes.}\
	\begin{enumerate}[label = \alph*)]
		\item The {\em margin} $\delta$ of a set of points $X \subseteq \{0,1\}^n$
		labeled by a function $f$ is defined as follows.  $$\delta = \max_{w \in
			\R^n, \|w\|_2 = 1,\ \theta \in \R} \{ \delta' \mid \forall x \in
		X,\ |w\cdot x - \theta| \geq \delta'  \text{ and }  (w \cdot x \geq \theta
		\text{ iff } f(x) = 1)  \}$$ (This is the $\delta$ of the
		Perceptron algorithm.) Give a set of $O(n)$ examples on $\{0,1\}^n$ that are
		linearly separable but for which the margin is exponentially small in $n$.
		\item Give an example of a linear threshold function on which the Perceptron algorithm can make exponentially many mistakes.
	\end{enumerate}

\textbf{Part (a):} To find examples that are linearly separable but have exponentially small margin, I'll create two classes of points in $\{0,1\}^n$. The classification rule $f$ will be such that a point $x \in \{0,1\}^n$ has $f(x)=1$ if and only if the first non-zero element appears at an odd-index position. The first class of points we'll call $\{a_1, a_2, \ldots, a_n\}$ where $a_i$ has its first $i-1$ components as $0$ and all remaining components as $1$. Mathematically this can be described as:
\begin{align}
	a_1 &= (1,1,\ldots,1) \\
	a_2 &= (0,1,1,\ldots,1) \\
	&\vdots \\
	a_n &= (0,0,\ldots,0,1)
\end{align}
The second class of points will be $\{b_1, b_2, \ldots, b_n\}$ where $b_i$ has its first $i-1$ components as $0$ followed by alternating $0$'s and $1$'s. Mathematically this can be described as:
\begin{align}
	b_1 &= (1,0,1,0,\ldots) \\
	b_2 &= (0,1,0,1,\ldots) \\
	&\vdots \\
	b_n &= (0,0,\ldots,0,1)
\end{align}
According to our classification rule, points $a_i$ and $b_i$ are labeled $1$ when $i$ is odd and $0$ when $i$ is even. To correctly classify these points, any weight vector $w$ must have $w_i > 0$ when $i$ is odd and $w_i < 0$ when $i$ is even. And specifically for the $b_i$ points, the magnitude of $w_i$ must dominate the sum of all subsequent weights: $|w_i| > \sum_{j>1} |w_j|$. These constraints on $w$ force it to decay exponentially. A valid weight vector would be $w = (2^{-1}, 2^{-2}, 2^{-3}, \ldots)$. Then, after normalizing the vector to unit length, the margin becomes $\delta = O(2^{-n})$ which is exponentially small in $n$.

\textbf{Part (b):} The Perceptron algorithm's mistake bound is inversely proportional to the square of the margin. Consider the weight vector $w = \left(1, -\frac{1}{2}, \frac{1}{4}, -\frac{1}{8}, \ldots, \frac{(-1)^{n-1}}{2^{n-1}}, 0\right)$. When normalized to unit length, this defines a linear threshold function $f(x) = 1$ if $w \cdot x \geq \theta$ and $f(x) = 0$ otherwise, where $\theta = 0$. For this function, the weights alternate signs and decrease exponentially in magnitude. The margin $\delta_w = \min_{x \in X} |w \cdot x|$ is less than $2^{-n}$, and the squared norm $\|w\|_2 = \sum_i w_i^2$ converges to a constant. By the Perceptron mistake bound theorem, the number of mistakes is bounded by $\frac{\|w\|^2}{\delta_2^2}$ which is at least $2^{2n}$ for this weight vector. Thus, the Perceptron algorithm can make exponentially many mistakes on this linear threshold function.
\end{problem}

\begin{problem} [10pts] \textbf{Logarithmic Mistake Bounds.} \quad
	\begin{enumerate}[label=\alph*)]
		\item Let $\C_n$ be a finite concept class.  Give a learning
		algorithm for $\C_n$ which has mistake bound $\log_2|\C_n|.$  Your algorithm
		need not be computationally efficient.
		\item Show a concept class $\C$ for which there exists an algorithm with a mistake bound that is asymptotically better than $\log_2|\C_n|$.
	\end{enumerate}

\textbf{Part (a):} \textbf{Algorithm $\mathcal{L}$:} 
\begin{enumerate}
	\item Initialize $H = \mathcal{C}_n$ (all possible hypotheses)
	\item For each input $x$:
		\begin{enumerate}
			\item Compute $h(x)$ for each $h \in H$
			\item Predict the majority vote of all hypotheses in $H$
			\item If the prediction is incorrect, eliminate all hypotheses in $H$ that made the same mistake
		\end{enumerate}
\end{enumerate}
Initially $|H| = |\mathcal{C}_n|$ but after each mistake, at least half of the hypotheses in $H$ are eliminated (as we predict with the majority so if a mistake is made, all hypotheses that voted with the majority were incorrect). The target concept always remains in $H$ as we never eliminate the correct hypothesis. Thus, since $|H|$ starts at $|\mathcal{C}_n|$ and is reduced by at least half after each mistake and $|H| \geq 1$ always as the target concept always remains in $H$, the maximum number of mistakes is bounded by $\log_2|\mathcal{C}_n|$.

\textbf{Part (b):} Consider the concept class $\mathcal{C}_n$ that represents labelings of vertices  on a regular polygon with a multiple of $3$ number of vertices (triangle, hexagon, nonagon, etc.) where each vertex is labeled with $0$ or $1$ and valid concepts only assign label $1$ to vertices that form a contiguous segment of exactly $2n$ connected neighboring points. Altogether, this concept class has exactly $3n$ possible concepts (one for each possible starting position of the $2n$ segment around the polygon). When using the halvign algorithm for this concept class, for any input vertex, exactly $2n$ out of the $3n$ concepts will label it as $1$. So if the algorithm makes a mistake on a prediction, it must eliminate either all concepts that predicted $1$ if the true label was $0$ which eliminates $\frac{2n}{3n}=\frac{2}{3}$ of the concepts, or all concepts that predicted $0$ if the true label was $1$ which eliminates $\frac{n}{3n}=\frac{1}{3}$ of the concepts. Therefore, in the worst case $2/3$ of the remaining hypotheses are eliminated with each mistake, bounding the number of mistakes by $\log_{3} |\mathcal{C}_n|$. This is asymptotically better than $\log_2 |\mathcal{C}_n|$.

\end{problem}

\begin{problem}[15 points] \textbf{Robust Winnow Algorithm.}
	In this question we examine the error robustness of the Winnow algorithm. Let $f$ be a monotone disjunction of $k$ variables. Suppose that an adversary manipulates the labels of some of the examples seen by the Winnow algorithm as it runs on $f$: during the online learning process, for some inputs $x$, the Winnow algorithm is told that the label of $x$ is $1-f(x)$ when the correct value was actually $f(x)$. When $f(x) = 1$ but the adversary flips its reported label to 0, we call this is a \emph{false negative example}. When $f(x) = 0$ but the algorithm is presented with the label 1, this is called a \emph{false positive example}.
	
Assume that the algorithm sees $s$ false negative examples and $t$ false positive examples during the course of its learning. Prove that the total number of mistakes the Winnow algorithm\footnote{if modified appropriately; see the note at the end of the problem statement.} will make (on examples that are labeled correctly) is $O(k \log{n} + ks + t)$.
\end{problem}
\begin{note}
	You will have to modify the Winnow algorithm to get this result. In class
	the demotion step set all weights $w_i = 0$ if $x_i = 1$. Argue why this will
	not work in the above setting and show that if instead you set $w_i \leftarrow
	w_i/2$, you will get the required bounds.
\end{note}

In the standard Winnow algorithm, when a false positive occurs, the weights of active variables are set to $0$ ($w_i = 0$ if $x_i = 1$). While this works well in noise-free environments, it's problematic with adversarial noise because a single false positive example would cause the algorithm to make no mistake but incorrectly demote weights, or worse, a false negative example would cause the algorithm to set the weights of relevant variables to $0$. Once a relevant variable's weight is set to $0$, it remains $0$ forever essentially removing that variable from consideration. With adversarial noise, this could permanently eliminate relevant variables from the hypothesis, making it impossible to learn the target concept. Instead, I'll use a modified version where during demotion, weights are halved rather than zeroed.
\begin{itemize}
	\item Initialize all weights $w_i = 1$
	\item Predict $1$ if $\sum_{i=1}^{n} w_i x_i \geq \theta$ where $\theta$ is the threshold
	\item If false negative, i.e., predict 0 when true label is 1, then $w_i \rightarrow 2w_i$ for all $i$  where $x_i=1$
	\item If false positive, i.e., predict 1 when true label is 0, then $w_i \rightarrow w_i/2$ for all $i$ where $x_i=1$
\end{itemize}

Mistakes on correctly labeled examples (normal learning) is one source of mistakes. In this case, each relevant variable can be promoted at most $\log n$ times since weights start at 1. As there are $k$ relevant variables in the disjunction, this results in at most $k \log n$ mistakes from normal learning. In false negative examples, weights of relevant variables can be incorrectly halved. When this happens, it might require a promotion to become relevant again. In the worst case, all $k$ relevant variables might need to be promoted again, so for each of the $s$ false negative examples, there's a potential amplification of $k$ mistakes, resulting in at most $k \cdot s$ additional mistakes. For false positives, the weights of irrelevant variables are mistakenly increased. Each such example can lead to at most one additional mistake when the algorithm encounters a correctly labeled negative example, so with $t$ false positive examples, we get at most $t$ additional mistakes. So altogether, we get $O(k \log n + ks + t)$ mistakes.


\begin{problem} [15pts] \textbf{VC Dimension of Linear Halfspaces.}
	Let $\C$ be the concept class of linear halfspaces in $\R^n$. A halfspace is specified by an inequality of the form $c(x) = \1[w \cdot x \geq \theta]$, where $w \in \R^n\setminus\{\mathbf{0}\}$ and $\theta \in \R$. Prove that the VC dimension of $\C$ is $n+1$.
\end{problem}
\begin{note}
	You may use the following result without proof:
	\begin{theorem}[\textbf{Radon's theorem}]
		Let $S =
		\{x^{(1)},\dots,x^{(m)}\} \subset \R^n$ be a set of $m$
		points in $\R^n.$  The {\em convex hull} of $S$ is the set
		\[
		\{ z \in \R^n \mid \exists \lambda_1,\ldots,\lambda_m \text{\ s.t.\ } z = \sum_{i=1}^m \lambda_i x^{(i)}, \text{\ with each\ }\lambda_i \geq 0 \text{\ and\ } \sum_{i=1}^m \lambda_i = 1\}
		\] 
		If $m \geq n+2$ then $S$ must have two disjoint subsets
		$S_1$ and $S_2$ whose convex hulls intersect.
	\end{theorem}
\end{note}

The steps to determining the VC dimension of linear halfspaces in $\mathbb{R}^n$ are:
\begin{enumerate}
	\item Showing that there exists a set of $n+1$ points that can be shattered by linear halfspaces
	\item Showing that no set of $n+2$ points can be shattered by linear halfspaces
\end{enumerate}

Consider any set $S$ of $n+2$ points in $\mathbb{R}^n$. By Radon's theorem, there exist disjoint subsets $S_1$ and $S_2$ whose convex hulls intersect. This means there exists a point $z$ that belongs to both convex hulls. Since $z$ is in both convex hulls, no linear halfspace can label all points in $S_1$ as 1 and all points in $S_2$ as 0 (or vice versa) as $z$ cannot be simultaneously labeled 0 and 1 which is a contradiction. Therefore, at least one dichotomy of $S$ cannot be realized by any linear halfspace which means $S$ cannot be shattered by linear halfspaces. Since this applies to any set of $n+2$ points, this shows that the VC dimension of linear halfspaces in $\mathbb{R}^n$ is at most $n+1$. Now to get the lower bound, consider a set of $n+1$ points that can be shattered by linear halfspaces. Specifically, let $S = \{s_0, s_1, \ldots, s_n\}$ where $s_0=\mathbf{0}$ and $s_i$ is the standard basis vector with a 1 in position $i$ and 0 everywhere else for $1 \leq i \leq n$. Now, take an arbitrary partition of $S$ into two sets $U$ and $V$ where we want to label points in $U$ as 1 and points in $V$ as 0. If either $U$ or $V$ is empty, the partition is trivially realizable, so let's assume we have a non-trivial case where they're both not empty. The key insight is that the convex hull of any subset of $S$ has a specific structure, namely, if the origin $s_0$ is not in the subset, the convex hull consists of poitns where only the components corresponding to the basis vectors in the subset can be non-zero. So for any non-trivial partition, we can always find a separating hyperplane: if $s_0 \in U$, define $w$ so that $w_i > 0$ for basis vectors $s_i \in U$ and $w_i < 0$ for basis vectors $s_i \in V$ and $\theta = 0$; if $s_0 \in V$ define $w$ so that $w_i > 0$ for basis vectors $s_i \in V$ and $w_i < 0$ for basis vectors $s_i \in U$ and $\theta = 0$. This ensures that the linear halfspace defined by $w \cdot x \geq \theta$ will correctly classify all points according to the partition. Since any partition of $n+1$ points in $S$ can be realized by some linear halfspace, $S$ can be shattered, proving the VC dimension is at least $n+1$. So, combining the two bounds we derived, the VC dimension of the concept class of linear halfspaces in $\mathbb{R}^n$ is exactly $n+1$.
\end{document}