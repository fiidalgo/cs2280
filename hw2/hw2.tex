\documentclass[11pt]{article}
\usepackage[margin=2.5cm]{geometry}
\usepackage{parskip}
\usepackage{microtype}
\usepackage{enumitem}
\usepackage{amsmath, amsfonts, amssymb, amsthm}
\usepackage{thmtools}
\usepackage{bbm}

\declaretheorem{theorem}
\declaretheorem[style=definition]{problem}
\declaretheorem[style=remark, numbered=no]{hint}
\declaretheorem[style=remark, numbered=no]{note}

\newcommand*{\C}{{\mathcal C}}
\newcommand*{\Hp}{{\mathcal H}}
\newcommand*{\R}{\mathbb{R}}
\newcommand*{\Z}{\mathbb{Z}}

\DeclareMathOperator{\E}{\mathbb{E}}
\DeclareMathOperator{\1}{\mathbbm{1}}
\DeclareMathOperator{\Ber}{Ber}
\DeclareMathOperator{\size}{size}
\DeclareMathOperator{\VCD}{VCD}


\begin{document}

\begin{flushright}
			Leslie G. Valiant \\
			TFs: Aayush Karan \& Kevin Cong
\end{flushright}

\begin{center}
\textbf{CS 228: Computational Learning Theory} \\
\textbf{Homework 2}\\ \textbf{Due: March 12, 11:59PM}
\end{center}

\textbf{Policy reminders:} You are strongly encouraged to type your solutions using LATEX. You may
discuss problems with your classmates, but not merely copy each others solutions. You must write
all solutions by yourself, list your collaborators on your problem sets and also appropriately cite
any resources outside of the class materials that you have used. You are not allowed to look up
solutions to the problems. Please do not use LLMs or LLM-assisted tools for finding solutions to the
problems. 
\rule{\linewidth}{0.4pt}

\begin{problem} [10pts] \textbf{Efficient Evaluability is Necessary.}
The definition of learning in the PAC model requires that hypotheses produced by a learning algorithms are evaluatable efficiently (that is, in polynomial time). In this problem you will show that this requirement is necessary for a meaningful notion of learning.

Suppose we consider relaxing this restriction, and let $\Hp$ be the class of all Turing machines (not necessarily polynomial time). Show that if $\C_n$ is the class of all Boolean circuits of size at most $p(n)$ for some fixed polynomial $p(\cdot)$, then $\C$ is efficiently PAC learnable using $\Hp$.

Argue that your solution shows that this relaxation trivializes the model of learning. (Hint: How could you use your solution to trivially learn other concept classes we've seen in class?)
	
\end{problem}
\begin{problem} [15pts] \textbf{Infinite Mistake-Bounded Model.}
	The \textit{Infinite Mistake-Bounded (IMB) model} is a Boolean on-line learning model where the number
of attributes in the world may be infinite. It is assumed, however, that each example has a finite
number of attributes assigned 1. An example is presented to the learner as a list of its positive
attributes. The learning scenario is the same as in the (standard) mistake-bounded model.

Provide an on-line learning algorithm that makes at most $\mathcal O(k \log n)$ mistakes when learning any
monotone disjunction of $k$ literals, if every example presented to the learner has at most $n$ positive
attributes.

\textit{Note: Assume for full credit that the learner does \textit{not} know the value of $n$ in advance. For partial credit, you can use the value of $n$ in your algorithm}
\end{problem}

\begin{problem} [15 points] \textbf{Function Classes: Linear Thresholds, DNFs, and Decision Lists.} \
	\begin{enumerate}[label=\alph*)]
		\item Prove that any 1-decision list over $x_1,\dots,x_n$ can be expressed as a linear threshold function $\1[w \cdot x \geq \theta]$, where $w \in \R^n\setminus\{\mathbf{0}\}$, $\theta \in \R$.
		\item Prove that the class of linear threshold functions and the class of
		poly$(n)$-term DNF are incomparable in that neither class is contained in the
		other.
	\end{enumerate}
\end{problem}
\begin{problem}[15 points] \textbf{Perceptrons Can Make Exponentially Many Mistakes.}\
	\begin{enumerate}[label = \alph*)]
		\item The {\em margin} $\delta$ of a set of points $X \subseteq \{0,1\}^n$
		labeled by a function $f$ is defined as follows.  $$\delta = \max_{w \in
			\R^n, \|w\|_2 = 1,\ \theta \in \R} \{ \delta' \mid \forall x \in
		X,\ |w\cdot x - \theta| \geq \delta'  \text{ and }  (w \cdot x \geq \theta
		\text{ iff } f(x) = 1)  \}$$ (This is the $\delta$ of the
		Perceptron algorithm.) Give a set of $O(n)$ examples on $\{0,1\}^n$ that are
		linearly separable but for which the margin is exponentially small in $n$.
		\item Give an example of a linear threshold function on which the Perceptron algorithm can make exponentially many mistakes.
	\end{enumerate}
\end{problem}

\begin{problem} [10pts] \textbf{Logarithmic Mistake Bounds.} \quad
	\begin{enumerate}[label=\alph*)]
		\item Let $\C_n$ be a finite concept class.  Give a learning
		algorithm for $\C_n$ which has mistake bound $\log_2|\C_n|.$  Your algorithm
		need not be computationally efficient.
		\item Show a concept class $\C$ for which there exists an algorithm with a mistake bound that is asymptotically better than $\log_2|\C_n|$.
	\end{enumerate}
\end{problem}

\begin{problem}[15 points] \textbf{Robust Winnow Algorithm.}
	In this question we examine the error robustness of the Winnow algorithm. Let $f$ be a monotone disjunction of $k$ variables. Suppose that an adversary manipulates the labels of some of the examples seen by the Winnow algorithm as it runs on $f$: during the online learning process, for some inputs $x$, the Winnow algorithm is told that the label of $x$ is $1-f(x)$ when the correct value was actually $f(x)$. When $f(x) = 1$ but the adversary flips its reported label to 0, we call this is a \emph{false negative example}. When $f(x) = 0$ but the algorithm is presented with the label 1, this is called a \emph{false positive example}.
	
Assume that the algorithm sees $s$ false negative examples and $t$ false positive examples during the course of its learning. Prove that the total number of mistakes the Winnow algorithm\footnote{if modified appropriately; see the note at the end of the problem statement.} will make (on examples that are labeled correctly) is $O(k \log{n} + ks + t)$.
\end{problem}
\begin{note}
	You will have to modify the Winnow algorithm to get this result. In class
	the demotion step set all weights $w_i = 0$ if $x_i = 1$. Argue why this will
	not work in the above setting and show that if instead you set $w_i \leftarrow
	w_i/2$, you will get the required bounds.
\end{note}


\begin{problem} [15pts] \textbf{VC Dimension of Linear Halfspaces.}
	Let $\C$ be the concept class of linear halfspaces in $\R^n$. A halfspace is specified by an inequality of the form $c(x) = \1[w \cdot x \geq \theta]$, where $w \in \R^n\setminus\{\mathbf{0}\}$ and $\theta \in \R$. Prove that the VC dimension of $\C$ is $n+1$.
\end{problem}
\begin{note}
	You may use the following result without proof:
	\begin{theorem}[\textbf{Radon's theorem}]
		Let $S =
		\{x^{(1)},\dots,x^{(m)}\} \subset \R^n$ be a set of $m$
		points in $\R^n.$  The {\em convex hull} of $S$ is the set
		\[
		\{ z \in \R^n \mid \exists \lambda_1,\ldots,\lambda_m \text{\ s.t.\ } z = \sum_{i=1}^m \lambda_i x^{(i)}, \text{\ with each\ }\lambda_i \geq 0 \text{\ and\ } \sum_{i=1}^m \lambda_i = 1\}
		\] 
		If $m \geq n+2$ then $S$ must have two disjoint subsets
		$S_1$ and $S_2$ whose convex hulls intersect.
	\end{theorem}
\end{note}
\end{document}