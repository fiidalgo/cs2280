\documentclass[11pt]{article}
\usepackage[margin=2.5cm]{geometry}
\usepackage{parskip}
\usepackage{microtype}
\usepackage{enumitem}
\usepackage{amsmath, amsfonts, amssymb, amsthm}
\usepackage{thmtools}
\usepackage{bbm}

\declaretheorem{theorem}
\declaretheorem[style=definition]{problem}
\declaretheorem[style=remark, numbered=no]{hint}
\declaretheorem[style=remark, numbered=no]{note}

\newcommand*{\C}{{\mathcal C}}
\newcommand*{\Hp}{{\mathcal H}}
\newcommand*{\R}{\mathbb{R}}
\newcommand*{\Z}{\mathbb{Z}}

\DeclareMathOperator{\E}{\mathbb{E}}
\DeclareMathOperator{\1}{\mathbbm{1}}
\DeclareMathOperator{\Ber}{Ber}


\begin{document}

\begin{flushright}
			Leslie G. Valiant \\
			TFs: Aayush Karan and Kevin Cong
\end{flushright}

\begin{center}
\textbf{CS 2280: Computational Learning Theory} \\
Homework 3 Solutions\\
\textbf{Due: Mar. 28, 11:59pm}
\end{center}

\paragraph{Policy reminders} You are strongly encouraged to type your solutions using
\LaTeX. You may discuss problems with your classmates, but not merely copy each others solutions. You must write all solutions by yourself, list your collaborators on your problem sets and also appropriately cite any resources outside of the class materials that you have used. You are not allowed to look up solutions to the problems. Please do not use LLMs or LLM-assisted tools for finding solutions to the problems.

\rule{\linewidth}{0.4pt}

\begin{problem}(10pt) \textbf{Saturating Sauer-Shelah.} Show that for any $d$ there is a concept class ${\cal C}$ of VC dimension $d$
such that for any $m$ there exists a set $S$ of $m$ points such that $|\Pi_{\cal
C}(S)|=\Phi_d(m).$
\end{problem}

\noindent

\paragraph{Sauer--Shelah Lemma.}
The Sauer--Shelah Lemma states that for a concept class $\mathcal{C}$ of VC dimension $d$, and any finite set $S$ of $m$ points,
\[
\lvert \Pi_{\mathcal{C}}(S)\rvert \;\le\; \Phi_d(m) \;=\; \sum_{i=0}^{d}\binom{m}{i}.
\]
We want to exhibit, for each fixed $d$, a concept class whose number of dichotomies exactly matches $\Phi_d(m)$ for every subset $S$ of size $m$. In other words, we seek a class $\mathcal{C}$ of VC dimension $d$ so that
\[
\bigl\lvert \Pi_{\mathcal{C}}(S)\bigr\rvert \;=\;\Phi_d(m)\quad\text{for all $m$ and all $S$ with $\lvert S\rvert = m$.}
\]

\paragraph{Construction of the saturating class.}
Let $X$ be an infinite domain (or at least sufficiently large). Define
\[
\mathcal{C} \;=\; \bigl\{\;c_T \;:\; T\subseteq X,\;\lvert T\rvert \le d\;\bigr\},
\]
where each ``concept'' $c_T$ is the indicator function of the finite set $T$, that is,
\[
c_T(x) \;=\;
\begin{cases}
1, & x\in T,\\
0, & x\notin T.
\end{cases}
\]
In other words, $\mathcal{C}$ consists of all subsets of $X$ whose size is at most $d$, interpreted as characteristic functions.

\paragraph{Why $\mathcal{C}$ has VC dimension $d$.}
First observe that no set of size larger than $d$ can be shattered, since each concept in $\mathcal{C}$ picks out at most $d$ points.  Hence
\[
\text{VC-dim}(\mathcal{C}) \;\le\; d.
\]
On the other hand, any set $S^*$ of size $d$ is shattered by $\mathcal{C}$: for any chosen subset $A\subseteq S^*$ (of any size up to $d$), we can take the concept $c_{A}$ to pick out exactly $A$ from $S^*$.  This shows
\[
\text{VC-dim}(\mathcal{C}) \;\ge\; d.
\]
Thus $\text{VC-dim}(\mathcal{C})=d$.

\paragraph{Why $\lvert \Pi_{\mathcal{C}}(S)\rvert = \Phi_d(\lvert S\rvert)$.}
Take any set $S\subseteq X$ of size $m$.  Then
\[
\Pi_{\mathcal{C}}(S)
\;=\;
\{\, c_T\!\restriction_{S}\;\mid\; T\in \mathcal{C}\}
\;=\;
\{\,T\cap S\;\mid\; |T|\le d\}.
\]
But $T\cap S$ is any subset of $S$ of size at most $d$.  Hence
\[
\bigl\lvert \Pi_{\mathcal{C}}(S)\bigr\rvert
\;=\;
\sum_{i=0}^d \binom{m}{i}
\;=\;
\Phi_d(m),
\]
exactly saturating the Sauer--Shelah bound for all $m$.  Therefore, for every set $S$ of $m$ points,
\[
\bigl\lvert \Pi_{\mathcal{C}}(S)\bigr\rvert
\;=\;
\Phi_d(m),
\]
as desired.

\paragraph{Conclusion.}
This family of ``indicator-of-$T$'' concepts thus has VC dimension $d$ and achieves the maximum possible number of dichotomies $\Phi_d(m)$ on every $m$-element subset $S$, thereby saturating the bound stated in the problem.



\begin{problem} (10pt) \textbf{Monotone Boolean functions are not PAC-learnable.} Let $x = x_1\dots x_n \in \{0,1\}^n$ and $y = y_1\dots y_n \in
\{0,1\}^n$ be two $n$-bit strings.  We say that $x \geq y$ if $x_i \geq y_i$ for
all $i.$ A boolean function $f: \{0,1\}^n \rightarrow \{0,1\}$ on variables
$x_1,\dots,x_n$ is said to be {\em monotone} if $x \geq y$ implies $f(x) \geq
f(y).$\\ Let $M_n$ denote the class of all monotone Boolean functions over
$\{0,1\}^n.$  Prove that there is no PAC learning algorithm for $M_n$ whose
running time is a polynomial function of $n, \frac{1}{\epsilon}, \frac{1}{\delta}$ (without
size$(c)$).
\end{problem}





\begin{problem} (10pt) \textbf{VC-dimension of parity functions.}
Define the class of parity functions ${\mathcal P}$ over $X = \{0, 1\}^n$ as
follows: Let $a \in \{0, 1\}^n$, then $\chi_a(x) = 1$ if $a \cdot x$ is odd and
$\chi_a(x) = 0$ otherwise, where $a \cdot x = \sum_{i=1}^n a_i x_i$. Prove that the VC-dimension of ${\mathcal P}$ is $n$. 

\end{problem}



\begin{problem} (10pt) \textbf{Compositional VC Dimension.}
Let $\mathcal{C}$ be a concept class over some domain $X$ and $\mathcal{F}_T$ be a concept class over $\{0, 1\}^T$. We define a class of functions $\mathcal{F}_T(\mathcal{C})$ over $X$ as follows. 
\[\mathcal{F}_T(\mathcal{C}) = \{g(c_1(x), c_2(x), . . . , c_T(x)) | g \in \mathcal{F}_T \hspace{0.25cm} \text{\rm and} \hspace{0.25cm} c_1, . . . , c_T \in C\}.\] 
Prove that VC-dim$(\mathcal{F}_T(\mathcal{C})) = O(\ell \log \ell)$ where $\ell =$ VC-dim$(\mathcal{F}_T) + T \cdot $VC-dim$(\mathcal{C})$. 
\end{problem}




%%%%%% PROBLEM %%%%%%
\begin{problem}(15pt) \textbf{Occam as a weak learning algorithm.} Let $\mathcal{C}$ be any concept class. Show that if there exists an $(\alpha, \beta)$-Occam algorithm for $\mathcal{C}$, then there exists an efficient randomized Occam algorithm that given sample $S$ of size $m$ for $c \in C$, with probability at least $1 - \delta$, outputs a hypothesis $h$ consistent with $S$ such that size$(h) \leq p(n,size(c), \log m)$ for some polynomial $p$.
\end{problem}

\begin{hint}
    You can assume here that the description of each real number that occurs in an execution of the Adaboost algorithm takes $O(1)$ space.
\end{hint}





%%%%%% PROBLEM %%%%%%

\begin{problem}(15pt) \textbf{Adaboost on weak learning algorithm.}  
Let $\C$ be a concept class and {\tt WeakLearn} be an algorithm that weakly PAC
learns $\C$ and generates hypotheses that have error of at most $1/2-\gamma$
for some positive $\gamma$ (assume for simplicity that {\tt WeakLearn} always
succeeds). Let $x$ be any point in the sample S of size $N \geq 2$.
\begin{enumerate}
\item Show that in the Adaboost algorithm, the error of hypothesis $h_t$ on
distribution $D_{t+1}$ is exactly $1/2.$
\item What is the maximum probability that the Adaboost algorithm can assign to
point $x$ in any of the boosting stages?
\item Assuming  that {\tt WeakLearn} fails, for as long as it possibly can, to
return the correct label for $x$, what is the maximum number of stages that it
will take the Adaboost algorithm to force  {\tt WeakLearn} to return a
hypothesis which is correct on $x$ (give the best upper bound you can). {\em You
can assume that initially every point has the same probability.}
\end{enumerate}
\end{problem}





%%%%%% PROBLEM %%%%%%



\end{document}